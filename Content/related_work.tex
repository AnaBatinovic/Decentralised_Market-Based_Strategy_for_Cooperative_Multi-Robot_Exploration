\section{RELATED WORK}

Algorithms for assigning robots to target points can be grouped into centralised and decentralised algorithms. 
Centralised task assignment for a multi-robot system may be less practical due to communication limits \cite{Dias2000}, robustness issues \cite{Dias2006}, or time required for algorithm execution and scalability \cite{Julia2012}. In the centralised approach, each mobile robot receives tasks assigned from a single central \emph{leader} using a centralised planning algorithm. During communication between the central leader and the mobile robots, information about the mobile robot poses is shared in order to perform real time mobile robot task allocation. The central leader may be a computer or a robot.

The centralised algorithm described in \cite{Burgard2005} improves robustness, but the negotiation process is complex because the coordination of mobile robots is performed by a central \emph{executive} which creates a global map and executes auction with the  bids from the mobile robots, while in the same time assigning tasks according to the received bids. The auctioneer thus offers a task and other mobile robots compete by bidding. The winner is the robot with the highest revenue (or inversely, with the lowest cost).

In a decentralised approach, the mobile robots are completely autonomous in the exploration process. Each mobile robot has its own local knowledge of the world and can decide its future actions by taking into account its current context and task, its own capacities and the capacities of the other mobile robots, through a negotiation process \cite{Yan2013}. Moreover, it usually has better reliability, flexibility, adaptability and robustness. 

One of the most representative decentralised approaches is a market model.
The general concept of the market-based approaches includes independence of the mobile robots in terms of planning, and the ability of mobile robots to take team resources into account. It is shown in \cite{Dias2003} when different team sizes are included, the decentralised market method has an advantage over the centralised approach in terms of travelled distance.

Authors in \cite{Zlot2002} use a market architecture for the multi-robot mapping and exploration problem that aims to maximise the difference between the benefit and the cost to achieve the same objective as in the proposed approach (minimise the overall exploration time). However, the proposed approach takes into account an occupancy function which makes mobile robots more dispersed in the environment. Furthermore, we introduce a decentralised market-based strategy that implements event-based communication between mobile robots, reducing the amount of information used in the negotiation process. Similarly to Michael et al. \cite{Michael2008}, we propose a strategy in which mobile robots are able to take part in negotiation for target point assignment with the assumption that the agents have knowledge of all target points, while we assume that only a single mobile robot can be assigned to a specific target point.  

We have also compared a centralised strategy based on the approach described in \cite{Burgard2005} that also takes into account cost and utility functions proposed in the next section with the proposed decentralised strategy based on market model. The central control agent in the centralised strategy implementation is a computer.

A popular approach for an unknown area exploration is the frontier based exploration algorithm introduced by Yamauchi \cite{Yamauchi1998}. The idea is to route mobile robots to the frontier. The frontier edges detection can be achieved using RRT (Rapidly Exploring Random Tree) algorithm by Umari and Mukhopadhyay \cite{Umari2017}. The RRT algorithm is biased towards unexplored regions and provides a general approach which can be extended to higher dimensional spaces. However, RRT algorithm proved not to be fast enough in instances when larger environments were explored, so we opted to use a dense frontier detection method \cite{Orsulic2019}.